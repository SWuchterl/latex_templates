%%% General settings

%%%%%%%%%%%%%%%%%%%%%%%%%%%%%%%%%%%%%%%%%%%%%%%%%%%%%%%%%%%%%%%%%%%%
%
%  Common definitions
%
%  N.B. use of \providecommand rather than \newcommand means
%       that a definition is ignored if already specified
%
%                                              L. Taylor 18 Feb 2005
%%%%%%%%%%%%%%%%%%%%%%%%%%%%%%%%%%%%%%%%%%%%%%%%%%%%%%%%%%%%%%%%%%%%

% Some shorthand
% turn off italics
\newcommand {\etal}{\mbox{et al.}\xspace} %et al. - no preceding comma
\newcommand {\ie}{\mbox{i.e.}\xspace}     %i.e.
\newcommand {\eg}{\mbox{e.g.}\xspace}     %e.g.
\newcommand {\etc}{\mbox{etc.}\xspace}     %etc.
\newcommand {\vs}{\mbox{\sl vs.}\xspace}      %vs.
\newcommand {\mdash}{\ensuremath{\mathrm{-}}} % for use within formulas

% some terms whose definition we may change
\newcommand {\Lone}{Level-1\xspace} % Level-1 or L1 ?
\newcommand {\Ltwo}{Level-2\xspace}
\newcommand {\Lthree}{Level-3\xspace}

% Some software programs (alphabetized)
\providecommand{\ACERMC} {\textsc{AcerMC}\xspace}
\providecommand{\ALPGEN} {{\textsc{alpgen}}\xspace}
\providecommand{\CHARYBDIS} {{\textsc{charybdis}}\xspace}
\providecommand{\CLHEP} {\textsc{clhep}\xspace}
\providecommand{\CMKIN} {\textsc{cmkin}\xspace}
\providecommand{\CMSIM} {{\textsc{cmsim}}\xspace}
\providecommand{\CMSSW} {{\textsc{cmssw}}\xspace}
\providecommand{\COBRA} {{\textsc{cobra}}\xspace}
\providecommand{\COCOA} {{\textsc{cocoa}}\xspace}
\providecommand{\COMPHEP} {\textsc{CompHEP}\xspace}
\providecommand{\EVTGEN} {{\textsc{evtgen}}\xspace}
\providecommand{\FAMOS} {{\textsc{famos}}\xspace}
\providecommand{\GARCON} {\textsc{garcon}\xspace}
\providecommand{\GARFIELD} {{\textsc{garfield}}\xspace}
\providecommand{\GEANE} {{\textsc{geane}}\xspace}
\providecommand{\GEANTfour} {{\textsc{geant4}}\xspace}
\providecommand{\GEANTthree} {{\textsc{geant3}}\xspace}
\providecommand{\GEANT} {{\textsc{geant}}\xspace}
\providecommand{\HDECAY} {\textsc{hdecay}\xspace}
\providecommand{\HERWIG} {{\textsc{herwig}}\xspace}
\providecommand{\HIGLU} {{\textsc{higlu}}\xspace}
\providecommand{\HIJING} {{\textsc{hijing}}\xspace}
\providecommand{\IGUANA} {\textsc{iguana}\xspace}
\providecommand{\ISAJET} {{\textsc{isajet}}\xspace}
\providecommand{\ISAPYTHIA} {{\textsc{isapythia}}\xspace}
\providecommand{\ISASUGRA} {{\textsc{isasugra}}\xspace}
\providecommand{\ISASUSY} {{\textsc{isasusy}}\xspace}
\providecommand{\ISAWIG} {{\textsc{isawig}}\xspace}
\providecommand{\LCG} {{\textsc{lcg}}\xspace}
\providecommand{\WLCG} {{\textsc{wlcg}}\xspace}
\providecommand{\MADGRAPH} {\textsc{madgraph}\xspace}
\providecommand{\MCATNLO} {\textsc{mc@nlo}\xspace}
\providecommand{\MCFM} {\textsc{mcfm}\xspace}
\providecommand{\MILLEPEDE} {{\textsc{millepede}}\xspace}
\providecommand{\ORCA} {{\textsc{orca}}\xspace}
\providecommand{\OSCAR} {{\textsc{oscar}}\xspace}
\providecommand{\PHOTOS} {\textsc{photos}\xspace}
\providecommand{\PROSPINO} {\textsc{prospino}\xspace}
\providecommand{\PYTHIA} {{\textsc{pythia}}\xspace}
\providecommand{\POWHEG} {{\textsc{powheg}}\xspace}
\providecommand{\ROOT} {{\textsc{root}}\xspace}
\providecommand{\SHERPA} {{\textsc{sherpa}}\xspace}
\providecommand{\TAUOLA} {\textsc{tauola}\xspace}
\providecommand{\TOPREX} {\textsc{TopReX}\xspace}
\providecommand{\XDAQ} {{\textsc{xdaq}}\xspace}


%  Experiments
\newcommand {\DZERO}{D\O\xspace}     %etc.


% Measurements and units...

\newcommand{\de}{\ensuremath{^\circ}}
\newcommand{\ten}[1]{\ensuremath{\times \text{10}^\text{#1}}}
\newcommand{\unit}[1]{\ensuremath{\text{\,#1}}\xspace}
\newcommand{\mum}{\ensuremath{\,\mu\text{m}}\xspace}
\newcommand{\micron}{\ensuremath{\,\mu\text{m}}\xspace}
\newcommand{\km}{\ensuremath{\,\text{km}}\xspace}
\newcommand{\m}{\ensuremath{\,\text{m}}\xspace}
\newcommand{\cm}{\ensuremath{\,\text{cm}}\xspace}
\newcommand{\mm}{\ensuremath{\,\text{mm}}\xspace}
\newcommand{\fm}{\ensuremath{\,\text{fm}}\xspace}
\newcommand{\mus}{\ensuremath{\,\mu\text{s}}\xspace}
\newcommand{\ns}{\ensuremath{\,\text{ns}}\xspace}
\newcommand{\eV}{\ensuremath{\,\text{e\hspace{-.08em}V}}\xspace}
\newcommand{\keV}{\ensuremath{\,\text{ke\hspace{-.08em}V}}\xspace}
\newcommand{\MeV}{\ensuremath{\,\text{Me\hspace{-.08em}V}}\xspace}
\newcommand{\GeV}{\ensuremath{\,\text{Ge\hspace{-.08em}V}}\xspace}
\newcommand{\TeV}{\ensuremath{\,\text{Te\hspace{-.08em}V}}\xspace}
\newcommand{\PeV}{\ensuremath{\,\text{Pe\hspace{-.08em}V}}\xspace}
\newcommand{\keVc}{\ensuremath{{\,\text{ke\hspace{-.08em}V\hspace{-0.16em}/\hspace{-0.08em}}c}}\xspace}
\newcommand{\MeVc}{\ensuremath{{\,\text{Me\hspace{-.08em}V\hspace{-0.16em}/\hspace{-0.08em}}c}}\xspace}
\newcommand{\GeVc}{\ensuremath{{\,\text{Ge\hspace{-.08em}V\hspace{-0.16em}/\hspace{-0.08em}}c}}\xspace}
\newcommand{\TeVc}{\ensuremath{{\,\text{Te\hspace{-.08em}V\hspace{-0.16em}/\hspace{-0.08em}}c}}\xspace}
\newcommand{\keVcc}{\ensuremath{{\,\text{ke\hspace{-.08em}V\hspace{-0.16em}/\hspace{-0.08em}}c^\text{2}}}\xspace}
\newcommand{\MeVcc}{\ensuremath{{\,\text{Me\hspace{-.08em}V\hspace{-0.16em}/\hspace{-0.08em}}c^\text{2}}}\xspace}
\newcommand{\GeVcc}{\ensuremath{{\,\text{Ge\hspace{-.08em}V\hspace{-0.16em}/\hspace{-0.08em}}c^\text{2}}}\xspace}
\newcommand{\TeVcc}{\ensuremath{{\,\text{Te\hspace{-.08em}V\hspace{-0.16em}/\hspace{-0.08em}}c^\text{2}}}\xspace}

\newcommand{\pbinv} {\mbox{\ensuremath{\,\text{pb}^\text{$-$1}}}\xspace}
\newcommand{\fbinv} {\mbox{\ensuremath{\,\text{fb}^\text{$-$1}}}\xspace}
\newcommand{\nbinv} {\mbox{\ensuremath{\,\text{nb}^\text{$-$1}}}\xspace}
\newcommand{\percms}{\ensuremath{\,\text{cm}^\text{$-$2}\,\text{s}^\text{$-$1}}\xspace}
\newcommand{\lumi}{\ensuremath{\mathcal{L}}\xspace}
\newcommand{\Lumi}{\ensuremath{\mathcal{L}}\xspace}%both upper and lower
%
% Need a convention here:
\newcommand{\LvLow}  {\ensuremath{\mathcal{L}=\text{10}^\text{32}\,\text{cm}^\text{$-$2}\,\text{s}^\text{$-$1}}\xspace}
\newcommand{\LLow}   {\ensuremath{\mathcal{L}=\text{10}^\text{33}\,\text{cm}^\text{$-$2}\,\text{s}^\text{$-$1}}\xspace}
\newcommand{\lowlumi}{\ensuremath{\mathcal{L}=\text{2}\times \text{10}^\text{33}\,\text{cm}^\text{$-$2}\,\text{s}^\text{$-$1}}\xspace}
\newcommand{\LMed}   {\ensuremath{\mathcal{L}=\text{2}\times \text{10}^\text{33}\,\text{cm}^\text{$-$2}\,\text{s}^\text{$-$1}}\xspace}
\newcommand{\LHigh}  {\ensuremath{\mathcal{L}=\text{10}^\text{34}\,\text{cm}^\text{$-$2}\,\text{s}^\text{$-$1}}\xspace}
\newcommand{\hilumi}  {\ensuremath{\mathcal{L}=\text{10}^\text{34}\,\text{cm}^\text{$-$2}\,\text{s}^\text{$-$1}}\xspace}

% Some usual physics terms

\newcommand{\zp}{\ensuremath{\mathrm{Z}^\prime}\xspace}
\newcommand{\brOf}[1]{\ensuremath{\mathcal{B}\!\left(#1\right)}}
\newcommand{\br}{\ensuremath{\mathcal{B}}\xspace}
\newcommand{\xsecOf}[1]{\ensuremath{\sigma(#1)}\xspace}
\newcommand{\xsec}{\ensuremath{\sigma}\xspace}

% SM (still to be classified)

\newcommand{\kt}{\ensuremath{k_{\mathrm{T}}}\xspace}
\newcommand{\BC}{\ensuremath{{B_{\mathrm{c}}}}\xspace}
\newcommand{\bbarc}{\ensuremath{{\overline{b}c}}\xspace}
\newcommand{\bbbar}{\ensuremath{{b\overline{b}}}\xspace}
\newcommand{\ccbar}{\ensuremath{{c\overline{c}}}\xspace}
\newcommand{\JPsi}{\ensuremath{{J}\hspace{-.08em}/\hspace{-.14em}\psi}\xspace}
\newcommand{\bspsiphi}{\ensuremath{B_s \to \JPsi\, \phi}\xspace}
%\newcommand{\ttbar}{\ensuremath{{t\overline{t}}}\xspace}
\newcommand{\AFB}{\ensuremath{A_\text{FB}}\xspace}
\newcommand{\EE}{\ensuremath{e^+e^-}\xspace}
\newcommand{\MM}{\ensuremath{\mu^+\mu^-}\xspace}
\newcommand{\TT}{\ensuremath{\tau^+\tau^-}\xspace}
\newcommand{\wangle}{\ensuremath{\sin^{2}\theta_{\text{eff}}^\text{lept}(M^2_\mathrm{Z})}\xspace}
\newcommand{\ttbar}{\ensuremath{{t\overline{t}}}\xspace}
\newcommand{\stat}{\ensuremath{\,\text{(stat.)}}\xspace}
\newcommand{\syst}{\ensuremath{\,\text{(syst.)}}\xspace}
% these moved to similar defs
%\newcommand{\Etmiss}{\ensuremath{E_{\mathrm{T}\!{\rm miss}}}}
%\newcommand{\VEtmiss}{\ensuremath{{\vec E}_{\mathrm{T}\!{\rm miss}}}}

%%%  E-gamma definitions
\newcommand{\HGG}{\ensuremath{\mathrm{H}\to\gamma\gamma}}
\newcommand{\gev}{\GeV}
\newcommand{\GAMJET}{\ensuremath{\gamma + \text{jet}}}
\newcommand{\PPTOJETS}{\ensuremath{\mathrm{pp}\to\text{jets}}}
\newcommand{\PPTOGG}{\ensuremath{\mathrm{pp}\to\gamma\gamma}}
\newcommand{\PPTOGAMJET}{\ensuremath{\mathrm{pp}\to\gamma +
\mathrm{jet}
}}
\newcommand{\MH}{\ensuremath{\mathrm{M_{\mathrm{H}}}}}
\newcommand{\RNINE}{\ensuremath{\mathrm{R}_\mathrm{9}}}
\newcommand{\DR}{\ensuremath{\Delta\mathrm{R}}}



% Physics symbols ...

\newcommand{\PT}{\ensuremath{p_{\mathrm{T}}}\xspace}
\newcommand{\pt}{\ensuremath{p_{\mathrm{T}}}\xspace}
\newcommand{\ET}{\ensuremath{E_{\mathrm{T}}}\xspace}
\newcommand{\HT}{\ensuremath{H_{\mathrm{T}}}\xspace}
\newcommand{\et}{\ensuremath{E_{\mathrm{T}}}\xspace}
\providecommand{\Em}{\ensuremath{E\hspace{-0.6em}/}\xspace}
\providecommand{\Pm}{\ensuremath{p\hspace{-0.5em}/}\xspace}
\providecommand{\PTm}{\ensuremath{{p}_\mathrm{T}\hspace{-1.02em}/}\xspace}
\providecommand{\PTslash}{\ensuremath{{p}_\mathrm{T}\hspace{-1.02em}/}\xspace}
\newcommand{\ETm}{\ensuremath{E_{\mathrm{T}}^{\text{miss}}}\xspace}
\newcommand{\MET}{\ensuremath{E_{\mathrm{T}}^{\text{miss}}}\xspace}
\newcommand{\ETmiss}{\ensuremath{E_{\mathrm{T}}^{\text{miss}}}\xspace}
\newcommand{\VEtmiss}{\ensuremath{{\vec E}_{\mathrm{T}}^{\text{miss}}}\xspace}

% roman face derivative
\providecommand{\dd}[2]{\ensuremath{\frac{\mathrm{d} #1}{\mathrm{d} #2}}}
%%%%%%
% From Albert
%

\newcommand{\ga}{\ensuremath{\gtrsim}}
\newcommand{\la}{\ensuremath{\lesssim}}
%\def\ga{\mathrel{\rlap{\raise.6ex\hbox{$>$}}{\lower.6ex\hbox{$\sim$}}}}
%\def\la{\mathrel{\rlap{\raise.6ex\hbox{$<$}}{\lower.6ex\hbox{$\sim$}}}}
%
\newcommand{\swsq}{\ensuremath{\sin^2\theta_W}\xspace}
\newcommand{\cwsq}{\ensuremath{\cos^2\theta_W}\xspace}
\newcommand{\tanb}{\ensuremath{\tan\beta}\xspace}
\newcommand{\tanbsq}{\ensuremath{\tan^{2}\beta}\xspace}
\newcommand{\sidb}{\ensuremath{\sin 2\beta}\xspace}
\newcommand{\alpS}{\ensuremath{\alpha_S}\xspace}
\newcommand{\alpt}{\ensuremath{\tilde{\alpha}}\xspace}

\newcommand{\QL}{\ensuremath{Q_L}\xspace}
\newcommand{\sQ}{\ensuremath{\tilde{Q}}\xspace}
\newcommand{\sQL}{\ensuremath{\tilde{Q}_L}\xspace}
\newcommand{\ULC}{\ensuremath{U_L^C}\xspace}
\newcommand{\sUC}{\ensuremath{\tilde{U}^C}\xspace}
\newcommand{\sULC}{\ensuremath{\tilde{U}_L^C}\xspace}
\newcommand{\DLC}{\ensuremath{D_L^C}\xspace}
\newcommand{\sDC}{\ensuremath{\tilde{D}^C}\xspace}
\newcommand{\sDLC}{\ensuremath{\tilde{D}_L^C}\xspace}
\newcommand{\LL}{\ensuremath{L_L}\xspace}
\newcommand{\sL}{\ensuremath{\tilde{L}}\xspace}
\newcommand{\sLL}{\ensuremath{\tilde{L}_L}\xspace}
\newcommand{\ELC}{\ensuremath{E_L^C}\xspace}
\newcommand{\sEC}{\ensuremath{\tilde{E}^C}\xspace}
\newcommand{\sELC}{\ensuremath{\tilde{E}_L^C}\xspace}
\newcommand{\sEL}{\ensuremath{\tilde{E}_L}\xspace}
\newcommand{\sER}{\ensuremath{\tilde{E}_R}\xspace}
\newcommand{\sFer}{\ensuremath{\tilde{f}}\xspace}
\newcommand{\sQua}{\ensuremath{\tilde{q}}\xspace}
\newcommand{\sUp}{\ensuremath{\tilde{u}}\xspace}
\newcommand{\suL}{\ensuremath{\tilde{u}_L}\xspace}
\newcommand{\suR}{\ensuremath{\tilde{u}_R}\xspace}
\newcommand{\sDw}{\ensuremath{\tilde{d}}\xspace}
\newcommand{\sdL}{\ensuremath{\tilde{d}_L}\xspace}
\newcommand{\sdR}{\ensuremath{\tilde{d}_R}\xspace}
\newcommand{\sTop}{\ensuremath{\tilde{t}}\xspace}
\newcommand{\stL}{\ensuremath{\tilde{t}_L}\xspace}
\newcommand{\stR}{\ensuremath{\tilde{t}_R}\xspace}
\newcommand{\stone}{\ensuremath{\tilde{t}_1}\xspace}
\newcommand{\sttwo}{\ensuremath{\tilde{t}_2}\xspace}
\newcommand{\sBot}{\ensuremath{\tilde{b}}\xspace}
\newcommand{\sbL}{\ensuremath{\tilde{b}_L}\xspace}
\newcommand{\sbR}{\ensuremath{\tilde{b}_R}\xspace}
\newcommand{\sbone}{\ensuremath{\tilde{b}_1}\xspace}
\newcommand{\sbtwo}{\ensuremath{\tilde{b}_2}\xspace}
\newcommand{\sLep}{\ensuremath{\tilde{l}}\xspace}
\newcommand{\sLepC}{\ensuremath{\tilde{l}^C}\xspace}
\newcommand{\sEl}{\ensuremath{\tilde{e}}\xspace}
\newcommand{\sElC}{\ensuremath{\tilde{e}^C}\xspace}
\newcommand{\seL}{\ensuremath{\tilde{e}_L}\xspace}
\newcommand{\seR}{\ensuremath{\tilde{e}_R}\xspace}
\newcommand{\snL}{\ensuremath{\tilde{\nu}_L}\xspace}
\newcommand{\sMu}{\ensuremath{\tilde{\mu}}\xspace}
\newcommand{\sNu}{\ensuremath{\tilde{\nu}}\xspace}
\newcommand{\sTau}{\ensuremath{\tilde{\tau}}\xspace}
\newcommand{\Glu}{\ensuremath{g}\xspace}
\newcommand{\sGlu}{\ensuremath{\tilde{g}}\xspace}
\newcommand{\Wpm}{\ensuremath{{W^{\pm}}}\xspace}
\newcommand{\sWpm}{\ensuremath{\tilde{W}^{\pm}}\xspace}
\newcommand{\Wz}{\ensuremath{{W^{0}}}\xspace}
\newcommand{\sWz}{\ensuremath{\tilde{W}^{0}}\xspace}
\newcommand{\sWino}{\ensuremath{\tilde{W}}\xspace}
\newcommand{\Bz}{\ensuremath{B^{0}}\xspace}
\newcommand{\sBz}{\ensuremath{\tilde{B}^{0}}\xspace}
\newcommand{\sBino}{\ensuremath{\tilde{B}}\xspace}
\newcommand{\Zz}{\ensuremath{{Z^{0}}}\xspace}
\newcommand{\sZino}{\ensuremath{\tilde{Z}^{0}}\xspace}
\newcommand{\sGam}{\ensuremath{\tilde{\gamma}}\xspace}
\newcommand{\chiz}{\ensuremath{\tilde{\chi}^{0}}\xspace}
\newcommand{\chip}{\ensuremath{\tilde{\chi}^{+}}\xspace}
\newcommand{\chim}{\ensuremath{\tilde{\chi}^{-}}\xspace}
\newcommand{\chipm}{\ensuremath{\tilde{\chi}^{\pm}}\xspace}
\newcommand{\Hone}{\ensuremath{H_{d}}\xspace}
\newcommand{\sHone}{\ensuremath{\tilde{H}_{d}}\xspace}
\newcommand{\Htwo}{\ensuremath{H_{u}}\xspace}
\newcommand{\sHtwo}{\ensuremath{\tilde{H}_{u}}\xspace}
\newcommand{\sHig}{\ensuremath{\tilde{H}}\xspace}
\newcommand{\sHa}{\ensuremath{\tilde{H}_{a}}\xspace}
\newcommand{\sHb}{\ensuremath{\tilde{H}_{b}}\xspace}
\newcommand{\sHpm}{\ensuremath{\tilde{H}^{\pm}}\xspace}
\newcommand{\hz}{\ensuremath{{h^{0}}}\xspace}
\newcommand{\Hz}{\ensuremath{{H^{0}}}\xspace}
\newcommand{\Az}{\ensuremath{{A^{0}}}\xspace}
\newcommand{\Hpm}{\ensuremath{{H^{\pm}}}\xspace}
\newcommand{\sGra}{\ensuremath{\tilde{G}}\xspace}
%
\newcommand{\mtil}{\ensuremath{\tilde{m}}\xspace}
%
\newcommand{\rpv}{\ensuremath{\rlap{\kern.2em/}R}\xspace}
\newcommand{\LLE}{\ensuremath{LL\bar{E}}\xspace}
\newcommand{\LQD}{\ensuremath{LQ\bar{D}}\xspace}
\newcommand{\UDD}{\ensuremath{\overline{UDD}}\xspace}
\newcommand{\Lam}{\ensuremath{\lambda}\xspace}
\newcommand{\Lamp}{\ensuremath{\lambda'}\xspace}
\newcommand{\Lampp}{\ensuremath{\lambda''}\xspace}
%
\newcommand{\spinbd}[2]{\ensuremath{\bar{#1}_{\dot{#2}}}\xspace}

\newcommand{\MD}{\ensuremath{{M_\mathrm{D}}}\xspace}% ED mass
\newcommand{\Mpl}{\ensuremath{{M_\mathrm{Pl}}}\xspace}% Planck mass
\newcommand{\Rinv} {\ensuremath{{R}^{-1}}\xspace}



%%%%%%%%%%%%%%%%%%%%%%%%%%%%%%%%%%%%%%%%%%%%%%%%%%%%%%%%%%%%%%%%%%%%
%
% Hyphenations (only need to add here if you get a nasty word break)
%
\hyphenation{en-viron-men-tal}%    just an example
 	% official CMS definitions
% units and symbols
% most of them are defined in ptdr-definitions.tex, the ones that are missing there are defined here
\newcommand{\pb}{\ensuremath{\,\text{pb}}\xspace}
\newcommand{\MV}{\ensuremath{\,\text{MV}}\xspace}
\newcommand{\MVm}{\ensuremath{\,\text{MV\hspace{-0.16em}/\hspace{-0.08em}m}}\xspace}
\newcommand{\MHz}{\ensuremath{\,\text{MHz}}\xspace}
\newcommand{\T}{\ensuremath{\,\text{T}}\xspace}
\newcommand{\mrad}{\ensuremath{\,\text{mrad}}\xspace}
\newcommand{\mt}{\ensuremath{m_{\mathrm{T}}}\xspace}
\newcommand{\metSlash}{\ensuremath{{\not\mathrel{E}}_\mathrm{T}}} % alternative version to \ETslash, w/o spacing problem

% period and comma after formulas with some extra spacing
\newcommand{\paf}{\ .}
\newcommand{\caf}{\ ,}
\newcommand{\waf}[1]{\ \text{#1}}

% references
\AtBeginDocument{ % hyperref redefines \ref at beginning of the document
	\let\oldref\ref
	%%% PubCom recommendations are given as comments
	%%% Currently, I do not follow PubCom recommendations
	\newcommand{\refChap}[1]{\hyperref[#1]{Chapter~\oldref*{#1}}}		% 'Chapter 3'
	\newcommand{\refSec} [1]{\hyperref[#1]{Section~\oldref*{#1}}}		% 'Section 3'
	\newcommand{\refApp} [1]{\hyperref[#1]{Appendix~\oldref*{#1}}}		% 'Appendix B'
	\newcommand{\refFig} [1]{\hyperref[#1]{Figure~\oldref*{#1}}}		% 'Fig. 3'; at beginning of sentence 'Figure 3'
	\newcommand{\refTab} [1]{\hyperref[#1]{Table~\oldref*{#1}}}			% 'Table 3'
	\newcommand{\refEq}  [1]{\hyperref[#1]{Equation~(\oldref*{#1})}}% 'Eq. (3)'; at beginning of sentence 'Equation (3)'
}

%% singlets and doublets (for SM table)
\newcommand{\doublet}[2]{$\begin{pmatrix} #1 \\ #2 \end{pmatrix}_{\mathrm{L}}$}
\newcommand{\lsinglet}[1]{$\begin{array}{c} #1_\mathrm{R}^\mathrm{-}\end{array}$}
\newcommand{\qsinglet}[2]{$\begin{array}{c} #1_\mathrm{R} \\ #2_\mathrm{R}\end{array}$}
\newcommand{\doubarrc}[2]{$\begin{array}{c} #1 \\ #2  \end{array}$}
\newcommand{\doubarrl}[2]{$\hspace{-2ex}\begin{array}{l} #1 \\ #2  \end{array}$}
\newcommand{\doubarrr}[2]{$\begin{array}{r} #1 \\ #2  \end{array}\hspace{-2ex}$}
\newcommand{\singarrl}[1]{$\hspace{-2ex}\begin{array}{l} #1  \end{array}$}
\newcommand{\singarrr}[1]{$\begin{array}{r} #1  \end{array}\hspace{-2ex}$}

% defined to be equal symbol :=
\newcommand*{\defeq}{\mathrel{\vcenter{\baselineskip0.5ex \lineskiplimit0pt
                     \hbox{\scriptsize.}\hbox{\scriptsize.}}}%
                     =}
										
% ########## Hyphenations ##########
% \hyphenation{ex-am-ple}
		% personal definitions

%%%%%%%%%%%%%%%  Title page %%%%%%%%%%%%%%%%%%%%%%%%
\documentclass[11pt,twoside, openright, a4paper, pdftex, tdr]{new-cms-tdr}
\usepackage[left=29mm,right=25mm,top=25mm,bottom=14mm,includeheadfoot]{geometry}%includehead

%\usepackage{geometry} % see geometry.pdf on how to lay out the page. There's lots.
%\geometry{a4paper} % or letter or a5paper or ... etc
% \geometry{landscape} % rotated page geometry

\usepackage[latin1]{inputenc}

%% add line numbers
%\usepackage[mathlines]{lineno} % add option pagewise for new line number per page
%\linenumbers

\usepackage{ifthen}
\newboolean{bdraft}
\setboolean{bdraft}{true} %%set comments and lipsums on and off 

% define some colors
\usepackage[dvipsnames]{xcolor}
\definecolor{lightblue}{rgb}{0.85,0.85,0.92}
\definecolor{gray}{gray}{0.6}
\usepackage{color}
\definecolor{RWTHblue}{RGB}{0,84,159}%RWTH blau
\definecolor{RWTHlightblue}{RGB}{142,186,229}%RWTH hellblau
\definecolor{darkblue}{rgb}{0,0,0.5}

% lorem ipsum blind text
\usepackage{lipsum}
\newcommand{\lorem}{ \ifthenelse{\boolean{bdraft}} {\textcolor{lightblue}{\lipsum}} {} }
\setlipsumdefault{1}

% wider lines in tables and arrays
\renewcommand{\arraystretch}{1.1}

% manage "ToDo"s in text
\usepackage{todo} 

% define comments
\newcommand{\comment}[1]{ \ifthenelse{\boolean{bdraft}} {\textcolor{RedOrange}{\{#1\}}} {} }

% Give numbers to deeper levels, and show them in the TOC
\ifthenelse{\boolean{bdraft}} {\setcounter{tocdepth}{4}} {}
\ifthenelse{\boolean{bdraft}} {\setcounter{secnumdepth}{4}} {}


% width of pictures (if 2 pictures next to each other)
\newcommand{\pairwidth}{.481\textwidth}

% format captions
\usepackage[margin=10pt,skip=8pt, format=plain]{caption}
\KOMAoption{captions}{tableheading, bottombeside}

% configure default position of figures and tables
%\makeatletter
%\renewcommand{\fps@figure}{htbp}
%\renewcommand{\fps@table}{htbp}
%\makeatother

% make tables look nicer
\usepackage{booktabs}

% definition of particle names
\usepackage{general/pennames-pazo}
%\usepackage{hepnames}  % nice particle names, incompatible with mathpazo math fonts

% bibliography support
\usepackage[numbers,sort&compress]{natbib}

% Feynman graphs
%\usepackage{feynmp}
% Automize calls to mpost in TeXnicCenter
% see http://latex-community.org/forum/viewtopic.php?f=31&t=16193
\DeclareGraphicsRule{*}{mps}{*}{}
\makeatletter
\def\endfmffile{%
\fmfcmd{\p@rcent\space the end.^^J%
end.^^J%
endinput;}%
\if@fmfio
\immediate\closeout\@outfmf
\fi
\ifnum\pdfshellescape=\@ne
\immediate\write18{mpost \thefmffile}%
\fi}
\makeatother

% links within document
\usepackage[%
colorlinks, % verwende farbige Links
linkcolor=black, % Linkfarbe ist RWTH blau
citecolor=RWTHlightblue, % Zitatfarbe ist RWTH blau
bookmarks, % erstelle Bookmarks der Links
bookmarksopen, % Bookmarks werden beim Öffnen des Dokumentes ebenfalls geöffnet
bookmarksopenlevel=2,
urlcolor=black, % Hyperlinks sind RWTH blau 
bookmarksnumbered, % Bookmarks sind nummeriert
pdfborder={0 0 0},
plainpages=false,
pdfpagelabels,
% draft  % Draft-Version
final  % Endversion
]{hyperref}

\hypersetup{%
%	plainpages=false,
%	pdfpagemode=Normal,%Keine Navigatorspalte
%	pdfview=FitH,%Standard-View f�r Link
%	pdfstartview=FitH,%Start-Ansicht FitH,FitV,...
%	pdfpagelayout=TwoColumnRight,%OneColumn,TwoColumnLeft,TwoColumnRight,SinglePage
	colorlinks=true, % false: boxed links; true: colored links
%	bookmarksopen=true,
%	bookmarksnumbered=true,
%	bookmarksopenlevel=2,
%	pdfmenubar=true,
%	pdfwindowui=true,
%	pdffitwindow=true,
	linkcolor=black,
%	linkcolor=black,
%	linkbordercolor=false,%Rahmenfarbe um Links (1 0 0)Leerzeichen wichtig(R G B)
	citecolor=black,
%	citecolor=black,
	urlcolor=black,
	filecolor=darkblue
}

% create glossary
% http://ftp.uni-erlangen.de/ctan/macros/latex/contrib/glossaries/glossaries-user.pdf
% first try of options which are not necessarily optimal
% 'acronym' to obtain list of acronyms independent from glossary
% 'acronym' and 'nomain' to obtain only list of acronyms, without glossary
% 'xindy' uses a perl script to properly sort entries, including e.g. greek letters
%\usepackage[xindy,nomain,acronym,toc]{glossaries}
% set style of glossary
% here: use 2 columns, as we expect short entries (mainly acronyms)
%\usepackage{glossary-mcols}
%\setglossarystyle{mcolindex}
% set style of acronyms
%\setacronymstyle{long-short}
%\makeglossaries  % ensure glossary files are created

